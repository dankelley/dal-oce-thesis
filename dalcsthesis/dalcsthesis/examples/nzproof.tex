% Version 2006-12-05
\documentclass[12pt]{dalcsthesis}

%Theorem
\usepackage{nzproof,theorem}
\newtheorem{theorem}{Theorem}
\newtheorem{proposition}{Proposition}
\newtheorem{lemma}{Lemma}
\newtheorem{definition}{Definition}
\newtheorem{observation}{Observation}
\newtheorem{corollary}{Corollary}
\newtheorem{problem}{Problem}

\begin{document}

\mcs  % options are \mcs, \macs, \mec, \mhi, \phd, and \bcshon
\title{The title}
\author{Noah Body}
\defenceday{1}
\defencemonth{November}
\defenceyear{2006}
\convocation{May}{2007}

% Use multiple \supervisor commands for co-supervisors.
% Use one \reader command for each reader.

\supervisor{D. Prof}
\reader{D. Odaprof}
\reader{A. External}

\nolistoftables
\nolistoffigures

\frontmatter

\begin{abstract}
This is a test document.
\end{abstract}

\begin{acknowledgements}
Thanks to all the little people who make me look tall.
\end{acknowledgements}

\mainmatter

\chapter{Introduction}

Get it done!  Use reference material by Lamport~\cite{latex-by-lamport} or
Gooses, Mittelback, and Samarin~\cite{latex-companion}.

\chapter{Doing It}

\section{Getting Ready}

Get all the parts that I need.  I can throw in a whole pile of terms like
preparation,
methodology,
forethought,
and
analysis
as examples for me to use in the future.

\section{Next Step}

Do it!

Of course, you have to have pictures to show how you did it to make people
understand things better.

Theorem~\ref{theorem-greatest} is a sample theorem.  You could equally well
make it a lemma, definition, proposition, observation, corollary, or problem.

\begin{theorem}

    \label{theorem-greatest}

    This is the greatest theorem!

\end{theorem}

\begin{pf}
    Trivial. Just admire it obvious correctness!  Sometimes you just have to 
    make things short enough that people believe the result right away.

    Then there are other times like this one where you're waiving your hands 
    wildly and hoping that nobody notices...but they always do.

    Ending here, if I leave a blank line before the endproof line then the 
    box appears on a separate line.  If I leave no line, like now, the box 
    comes at the end of this line.
\end{pf}

Or, you can do a proof sketch:

\begin{ps}
    Just waive your hands around and make it look convincing.
\end{ps}

Another alternative is to give a label to your proof, which can also be
done to proof sketches:

\begin{pf}[Theorem~\ref{theorem-greatest}]
This is a delated proof to a result stated earlier in the text.
\end{pf}

\chapter{Conclusion}

Did it!

\bibliographystyle{plain}
\bibliography{simple}

\end{document}
